\section{Descripción del proyecto}

\begin{frame}
    \begin{columns}[t]
        \begin{column}{.5\textwidth}
          \tableofcontents[sections={1-2},currentsection]
        \end{column}
        \begin{column}{.5\textwidth}
          \tableofcontents[sections={3-4},currentsection]
        \end{column}
    \end{columns}
\end{frame}

\subsection{Descripción del proyecto}
\begin{frame}
    El siguiente código a explicar es parte de una clase llamada "Moogle" que 
contiene un método estático llamado "Query". Este método toma una cadena 
de consulta como parámetro y devuelve un objeto SearchResult. 
\end{frame}
\begin {frame}
El código utiliza varios diccionarios y listas para realizar la búsqueda: 
\begin {itemize}
\item La variable "NombreDeDocumentos" 
\item La variable "DireccionDeDocumentos"
\item La lista"TodasPalabrasDeLosDocumentosSinRepetir"
\item Las listas "Debe", "NoDebe", y "Cerca"
\item La variable "Sugerencia" 
\item Las listas "CercaDos" y "snippet" 
\item El diccionario "Importancia"
\item El diccionario "Palabra"
\item El diccionario "ScoreDeDocumentos" 
\item El diccionario "PalabraQueMasSeRepitePorDocumentos"
\item El diccionario "CantidadDeDocumentosQueApareceUnaPalabra" 
\item El diccionario "TFxIDF"
\item El diccionario "Vocabulary"
\item La variable "scoreOrdenado"
\end {itemize}
\end{frame}
\begin{frame}
Posterior a las creación de las variables se implemento un método Rellenar que 
se encarga de como su nombre lo indica de completar estas listas y 
diccionarios. Se separa por partes para su mejor comprensión: 
\begin{itemize}
    \item Rellenar "Vocabulary" 
    Este código recorre una lista de directorios que contienen documentos y crea 
    un diccionario de vocabulario para cada documento. El diccionario contiene 
    palabras como claves y una lista de las posiciones en las que aparecen en el 
    documento como valores
    \item Rellenar nombre y dirección de documentos
     Este código crea un arreglo de strings llamado "NombreDeDocumentos" que 
    contiene los nombres de los documentos presentes en una lista de directorios
    \item Rellenar todas las palabras de los documentos en 
    "TodasPalabrasDeLosDocumentosSinRepetir"
    \item Rellenar "PalabraQueMasSeRepitePorDocumentos" 
    \item Rellenar "CantidadDeDocumentosQueApareceUnaPalabra" 
   
\end{itemize}
\end{frame}
\begin{frame}
\begin{itemize}
    \item Rellenar Diccionario "TFxIDFI" 
    \item Rellenar "ScoreDeDocumentos"
    \item "Snippet" El código recorre la lista de documentos ordenados por su puntaje y crea un 
    fragmento de texto para cada uno de ellos que contiene las palabras 
    importantes de la consulta y algunas palabras adicionales alrededor de esas 
    palabras. 
    \item "Score" 
    Este código es una función que calcula el puntaje de relevancia de un 
    documento en base a una consulta. Utiliza el método TF-IDF para calcular la 
    importancia de cada palabra
    \item "LimpiarPalabra"
    Este texto es una función que recibe una palabra y la limpia eliminando 
    cualquier carácter no deseado
    \item "HallarTFxIDF" 
    Este código es una función que calcula el valor de la fórmula TFxIDF para una 
    palabra
    \item "HallarTFxIDFDelQuery" 
    Este código es una función que recibe un texto y calcula el valor de la fórmula 
    TFxIDF
    
\end{itemize}
    
\end{frame}
\begin{frame}
    \begin{itemize}
        \item "sugerencia" 
        Esta función recibe un texto y devuelve una sugerencia de corrección para 
        cada palabra en el texto que no existe
        \item "distanciaL" 
        La función "distanciaL" implementa el algoritmo de distancia de Levenshtein.
        \item "busca1" 
        La función "busca1" verifica si el primer carácter de una cadena está en un 
        conjunto de operadores dados
        \item "busca2" 
     La función "busca2" itera a través de cada carácter en una cadena llamada 
    "palabra
        \item "Operadores"la función crea un arreglo de objetos "SearchItem" utilizando los 
        resultados de la función "scoreOrdenado", y los devuelve como un objeto 
        SearchResult junto con la variable "Sugerencia". 
    \end{itemize}
    
\end{frame}
