\section{Introducción}\label{intro}

\begin{frame}
    \begin{columns}[t]
        \begin{column}{.5\textwidth}
          \tableofcontents[sections={1-2},currentsection]
        \end{column}
        \begin{column}{.5\textwidth}
          \tableofcontents[sections={3-4},currentsection]
        \end{column}
    \end{columns}
\end{frame}


\section{Qué es Moogle?}
\begin{frame}{¿Qué es Moogle?}
\begin{center}
    Es una aplicación web, desarrollada con tecnología .NET Core 6.0, 
    específicamente usando Blazor como *framework* web para la interfaz gráfica, 
    y en el lenguaje C#.     
\end{center}

\end{frame}
\subsection{¿Qué objetivo tiene?}
\begin{frame}{¿Qué objetivo tiene?}

    Objetivo
\begin{itemize}
  \item Este proyecto tiene como objetivo buscar una consulta ingresada por el usuario 
  en un conjunto de archivos de texto (con extensión `.txt`) que estén en la 
  carpeta `Content`.   
 
\end{itemize}
\end{frame}
\begin{frame}{¿Cómo esta estructurado?}
    Estructura
\begin{itemize}
  \item `MoogleServer` es un servidor web que renderiza la interfaz gráfica y sirve los 
  resultados. 
  \item `MoogleEngine` es una biblioteca de clases donde está implementada la 
  lógica del algoritmo de búsqueda. 
  
 
\end{itemize}
\end{frame}
