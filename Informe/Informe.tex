\documentclass{article}
\usepackage{amsmath, amsthm, amssymb}
\begin{document}
\title{Moogle}
\author{Dayron Valdivia}
\date{Julio, 2023}
\maketitle

\section{Descripción del Proyecto }
Es una aplicación web, desarrollada con tecnología .NET Core 6.0, 
específicamente usando Blazor como *framework* web para la interfaz gráfica, 
y en el lenguaje C sharp. 
La aplicación está dividida en dos componentes fundamentales: 
- `MoogleServer` es un servidor web que renderiza la interfaz gráfica y sirve los 
resultados. 
- `MoogleEngine` es una biblioteca de clases donde está implementada la 
lógica del algoritmo de búsqueda. 
Este proyecto tiene como objetivo buscar una consulta ingresada por el usuario 
en un conjunto de archivos de texto (con extensión `.txt`) que estén en la 
carpeta `Content`.
El siguiente código a explicar es parte de una clase llamada "Moogle" que 
contiene un método estático llamado "Query". Este método toma una cadena 
de consulta como parámetro y devuelve un objeto SearchResult. 
\subsection{Diccionarios y variables }
El código utiliza varios diccionarios y listas para realizar la búsqueda:
\begin{itemize}
    \item La variable "NombreDeDocumentos" es un arreglo de cadenas que contiene 
    los nombres de los documentos en la carpeta "Content".
    \item La variable "DireccionDeDocumentos" es un arreglo de cadenas que contiene 
las rutas completas de los documentos en la carpeta "Content". 
    \item La lista "TodasPalabrasDeLosDocumentosSinRepetir" contiene todas las 
palabras únicas encontradas en los documentos en la carpeta "Content". 
    \item Las listas "Debe", "NoDebe", y "Cerca" contienen las palabras clave 
especificadas en la consulta del usuario. 
    \item La variable "Sugerencia" contiene una sugerencia de corrección ortográfica si 
la consulta del usuario contiene una palabra mal escrita. 
    \item Las listas "CercaDos" y "snippet" se utilizan para generar fragmentos de texto 
relevantes para la consulta del usuario. 
    \item El diccionario "Importancia" se utiliza para asignar una importancia a cada 
documento en función de su relevancia para la consulta del usuario. 
    \item El diccionario "Palabra" se utiliza para contar el número de veces que aparece 
cada palabra en los documentos. 
    \item El diccionario "ScoreDeDocumentos" se utiliza para almacenar el puntaje de 
relevancia de cada documento en función de su contenido y la consulta del 
usuario. 
    \item El diccionario "PalabraQueMasSeRepitePorDocumentos" se utiliza para 
almacenar el número de veces que aparece la palabra que más se repite en 
cada documento. 
    \item El diccionario "CantidadDeDocumentosQueApareceUnaPalabra" se utiliza 
para almacenar el número de documentos en los que aparece cada palabra. 
    \item El diccionario "TFxIDF" se utiliza para almacenar el puntaje TF-IDF de cada 
palabra en cada documento. 
    \item El diccionario "Vocabulary" se utiliza para almacenar la posición de cada 
palabra en cada documento. 
    \item La variable "scoreOrdenado" se utiliza para almacenar los documentos 
ordenados por su puntaje de relevancia. 


\end{itemize} 
 
\subsection{Método Rellenar }
Posterior a las creación de las variables se implemento un método Rellenar que 
se encarga de como su nombre lo indica de completar estas listas y 
diccionarios. Se separa por partes para su mejor comprensión:
\begin{itemize}
    \item Rellenar "Vocabulary" 
    Este código recorre una lista de directorios que contienen documentos y crea 
    un diccionario de vocabulario para cada documento. El diccionario contiene 
    palabras como claves y una lista de las posiciones en las que aparecen en el 
    documento como valores. 
    Primero, se crea un diccionario vacío para cada documento. Luego, se lee el 
    contenido del archivo y se divide en palabras utilizando el método Split(). Cada 
    palabra se limpia utilizando el método LimpiarPalabra() posteriormente 
    explicado. 
    Luego, se comprueba si la palabra ya está en el diccionario. Si es así, se 
    agrega la posición actual a la lista de posiciones. Si no, se agrega una nueva 
    entrada al diccionario con la palabra como clave y una lista que contiene la 
    posición actual como valor. 
    Finalmente, el diccionario se agrega al diccionario Vocabulary.
    \item  Rellenar "nombre y dirección de documentos"
    Este código crea un arreglo de strings llamado "NombreDeDocumentos" que 
    contiene los nombres de los documentos presentes en una lista de directorios. 
    Para crear los nombres de los documentos, se utiliza la variable 
    "DireccionDeDocumentos" que contiene la ruta completa del directorio en el 
    que se encuentran los documentos. 
    Se utiliza el método "Length" para obtener la longitud de la cadena 
    "Content" que es la ruta relativa del directorio donde se encuentran los 
    documentos. 
    Luego se utiliza el operador ".." para retroceder dos niveles en la estructura de 
    carpetas y obtener el nombre del documento. 
    El resultado es un arreglo de strings que contiene únicamente los nombres de 
    los documentos presentes en los directorios, sin la ruta completa.
    \item Rellenar todas las palabras de los documentos en 
    "TodasPalabrasDeLosDocumentosSinRepetir" 
    Este código recorre una lista de directorios (almacenados en la variable 
    "DireccionDeDocumentos") y para cada uno de ellos, lee el contenido del 
    archivo y lo divide en palabras utilizando el método "Split". 
    Luego, recorre cada una de las palabras obtenidas y las limpia utilizando la 
    función "LimpiarPalabra". 
    Finalmente, verifica si esa palabra ya se encuentra en la lista 
    "TodasPalabrasDeLosDocumentosSinRepetir" y, en caso contrario, la agrega a 
    la lista. 
    El resultado es una lista de todas las palabras únicas presentes en los 
    documentos. 
    \item Rellenar "PalabraQueMasSeRepitePorDocumentos" 
    Este código recorre una lista de direcciones de documentos (almacenados en 
    la variable "DireccionDeDocumentos") y para cada uno de ellos, inicializa un 
    valor de 0 en la lista "PalabraQueMasSeRepitePorDocumentos". 
    Luego, recorre cada una de las palabras obtenidas de la lista "Vocabulary[doc]" 
    y verifica si la cantidad de veces que aparece esa palabra es mayor a la que 
    está almacenada en "PalabraQueMasSeRepitePorDocumentos[doc]". 
    Si es así, actualiza el valor almacenado en 
    "PalabraQueMasSeRepitePorDocumentos[doc]" con la cantidad de veces que 
    aparece esa palabra. 
    El resultado es una lista que indica la palabra que más se repite en cada uno 
    de los documentos.
    \item Rellenar "CantidadDeDocumentosQueApareceUnaPalabra" 
    Este código recorre una lista de palabras obtenidas de todos los documentos 
    sin repetir (almacenados en la variable 
    "TodasPalabrasDeLosDocumentosSinRepetir") y para cada una de ellas, 
    inicializa un valor de 0 en la lista 
    "CantidadDeDocumentosQueApareceUnaPalabra". 
    Luego, recorre cada uno de los documentos almacenados en "Vocabulary" y 
    verifica si la palabra actual aparece en el documento. Si es así, incrementa un 
    contador "a". 
    Finalmente, almacena el valor de "a" en la lista 
    "CantidadDeDocumentosQueApareceUnaPalabra" con la palabra actual como 
    clave. 
    El resultado es una lista que indica la cantidad de documentos en los que 
    aparece cada una de las palabras obtenidas de los documentos sin repetir.
    \item Rellenar Diccionario "TFxIDFI" 
    El código recorre una lista de documentos almacenados en la variable 
    "DireccionDeDocumentos". Para cada documento, se crea un diccionario vacío 
    llamado "TFxIDFI". 
    Luego, para cada palabra en la lista 
    "TodasPalabrasDeLosDocumentosSinRepetir", se verifica si la palabra aparece 
    en el documento actual (almacenado en "Vocabulary[item]"). Si la palabra 
    aparece, se llama a la función "HallarTFxIDF" explicada posteriormente para 
    calcular su valor TFxIDF y se agrega al diccionario "TFxIDFI" con la palabra 
    como clave. Si la palabra no aparece, se agrega al diccionario con un valor de 
    0. 
    Finalmente, el diccionario "TFxIDFI" se agrega al diccionario "TFxIDF" con el 
    nombre del documento actual como clave. El resultado final es un diccionario 
    que contiene los valores TFxIDF para cada palabra en cada documento.
    \item Rellenar "ScoreDeDocumentos" 
    El código recorre una lista de documentos almacenados en la variable 
    "DireccionDeDocumentos" y calcula un puntaje para cada uno de ellos basado 
    en su similitud con una consulta dada. 
    Primero, se verifica si el documento contiene todas las palabras obligatorias de 
    la consulta y ninguna de las palabras que no deben estar presentes. Si el 
    documento no cumple con estos criterios, se le asigna un puntaje de 0 y se 
    pasa al siguiente documento. 
    Si el documento cumple con los criterios anteriores, se verifica si contiene 
    alguna palabra importante de la consulta. Si es así, se calcula su puntaje 
    usando la función "Score" explicada posteriormente y se multiplica por un factor 
    de importancia específico para esa palabra. Si el documento no contiene 
    ninguna palabra importante, se verifica si contiene una combinación específica 
    de dos palabras (almacenadas en la variable "elmtos") y si es así, se calcula un 
    puntaje basado en la distancia entre las apariciones de esas dos palabras en el 
    documento. 
    Finalmente, se ordenan los documentos por su puntaje y se crea un diccionario 
    "scoreOrdenado" que contiene los documentos ordenados de mayor a menor 
    puntaje.
\end{itemize} 
\subsection{Otros Métodos }
Ahora hablaremos sobre la implementación de otros métodos: 
\begin{itemize}
    \item "Snippet" 
    El código recorre la lista de documentos ordenados por su puntaje y crea un 
    fragmento de texto para cada uno de ellos que contiene las palabras 
    importantes de la consulta y algunas palabras adicionales alrededor de esas 
    palabras. 
    Primero, se recorre la lista de palabras importantes de la consulta ("Palabra") y 
    se verifica si el documento contiene cada una de ellas. Si es así, se crea un 
    fragmento de texto que contiene la palabra y algunas palabras adicionales 
    alrededor de ella (hasta 6 palabras a cada lado). Si el documento no contiene 
    alguna palabra importante, se omite esa palabra en el fragmento de texto. 
    Finalmente, se agrega el fragmento de texto para cada palabra importante al 
    fragmento final del documento ("finalSnippet") y se agrega "..." si quedan más 
    palabras importantes por agregar. 
    Este proceso se repite para cada documento en la lista ordenada y se devuelve 
    una lista de fragmentos de texto ("snippets") correspondientes a cada 
    documento.
    \item "Score" 
    Este código es una función que calcula el puntaje de relevancia de un 
    documento en base a una consulta. Utiliza el método TF-IDF para calcular la 
    importancia de cada palabra en el documento y en la consulta, y luego utiliza la 
    fórmula de similitud del coseno para calcular el puntaje de relevancia. 
    Primero, se divide el texto completo en palabras individuales y se limpian para 
    eliminar cualquier carácter no deseado. Luego, se recorre una lista de todas las 
    palabras en los documentos sin repetir y se calcula el puntaje TF-IDF para 
    cada palabra en el documento y en la consulta. 
    Finalmente, se utiliza la fórmula de similitud del coseno para calcular el puntaje 
    de relevancia del documento en base a la importancia de las palabras en la 
    consulta y en el documento. El resultado es un número entre 0 y 1, donde 1 
    indica que el documento es completamente relevante para la consulta y 0 
    indica que no es relevante en absoluto.
    \item "LimpiarPalabra" 
    Este texto es una función que recibe una palabra y la limpia eliminando 
    cualquier carácter no deseado al principio o al final de la palabra. Utiliza la 
    función char.IsLetterOrDigit para verificar si un carácter es una letra o un dígito, 
    y luego recorta la palabra para eliminar los caracteres no deseados. La función 
    devuelve la palabra limpia.
    \item "HallarTFxIDF" 
    Este código es una función que calcula el valor de la fórmula TFxIDF para una 
    palabra en un documento específico. La función recibe dos parámetros: la 
    palabra a calcular y el documento en el que se encuentra. 
    Primero, la función verifica si la palabra existe en la colección 
    CantidadDeDocumentosQueApareceUnaPalabra. Si la palabra existe, entonces 
    se procede a calcular los valores necesarios para la fórmula. 
    Los valores necesarios son: 
    - valor1: la cantidad de veces que la palabra aparece en el documento 
    específico. 
    - valor2: la cantidad de veces que la palabra que más se repite aparece en el 
    documento específico. 
    - valor3: la cantidad total de documentos en la colección. 
    - valor4: la cantidad de documentos en los que aparece la palabra. 
    Una vez que se tienen estos valores, se procede a calcular el valor de la 
    fórmula TFxIDF y se devuelve como resultado. Si la palabra no existe en la 
    colección, entonces se devuelve 0. 
    \item "HallarTFxIDFDelQuery" 
    Este código es una función que recibe un texto y calcula el valor de la fórmula 
    TFxIDF para cada palabra en el texto, en relación a una colección de 
    documentos. 
    Primero, se crea una lista llamada "textoEntero" que contiene todas las 
    palabras del texto sin caracteres vacíos ni especiales. Luego, se verifica si 
    cada palabra en "textoEntero" existe en un diccionario llamado "Palabra". Si la 
    palabra no existe en "Palabra", se agrega con un valor inicial de 1. Si la palabra 
    ya existe en "Palabra", se incrementa su valor en 1. 
    A continuación, se itera a través de cada palabra en una colección de todas las 
    palabras de los documentos sin repetir 
    ("TodasPalabrasDeLosDocumentosSinRepetir"). Para cada palabra, se verifica 
    si existe en "Palabra". Si la palabra existe, se calculan los valores necesarios 
    para la fórmula TFxIDF (valor1, valor2, valor3 y valor4) y se calcula el resultado 
    de la fórmula. Si la palabra no existe, se asigna un valor de 0 al resultado. 
    Finalmente, se devuelve un arreglo con los resultados de la fórmula TFxIDF 
    para cada palabra en la colección de palabras de los documentos sin repetir.
    \item "sugerencia" 
    Esta función recibe un texto y devuelve una sugerencia de corrección para 
    cada palabra en el texto que no existe en una colección de todas las palabras 
    de los documentos sin repetir ("TodasPalabrasDeLosDocumentosSinRepetir"). 
    Primero, se itera a través de cada palabra en un diccionario llamado "Palabra". 
    Si la palabra existe en la colección de palabras de los documentos sin repetir, 
    se agrega al resultado. Si la palabra no existe en la colección, se busca la 
    palabra más similar en la colección utilizando la distancia de Levenshtein 
    (función "distanciaL" explicada posteriormente) y se agrega la palabra más 
    similar al resultado. 
    La distancia de Levenshtein es una medida de la diferencia entre dos cadenas 
    de caracteres, es decir, el número mínimo de operaciones necesarias para 
    transformar una cadena en otra. En este caso, se utiliza para encontrar la 
    palabra más similar en la colección de palabras de los documentos sin repetir. 
    Finalmente, se devuelve el resultado con las sugerencias de corrección para 
    cada palabra en el texto que no existe en la colección de palabras de los 
    documentos sin repetir 
    \item "distanciaL" 
    La función "distanciaL" implementa el algoritmo de distancia de Levenshtein. 
    Toma dos cadenas de caracteres como entrada y devuelve el número mínimo 
    de operaciones necesarias para transformar una cadena en otra.
    \item "busca1" 
    La función "busca1" verifica si el primer carácter de una cadena está en un 
    conjunto de operadores dados. Devuelve true si lo encuentra y false si no lo 
    encuentra.
    \item "busca2" 
    La función "busca2" itera a través de cada carácter en una cadena llamada 
   "palabra". Si encuentra el carácter "~", devuelve true. Si no lo encuentra, 
   devuelve false. 
   \item "Operadores" 
   Este código es una función llamada "Operadores" que recibe como parámetro 
   una cadena de texto llamada "TextoCompleto". La función divide la cadena en 
   palabras utilizando el carácter de espacio en blanco como separador y las 
   almacena en un arreglo llamado "frase". 
   Luego, la función itera a través de cada palabra en el arreglo "frase" y verifica si 
   contiene alguno de los operadores especificados en el arreglo "operadores" 
   utilizando la función "busca1". Si encuentra un operador, la función extrae el 
   resto de la palabra a partir del índice donde se encuentra el operador y lo 
   almacena en una lista correspondiente (Debe, NoDebe o Importancia) según el 
   tipo de operador encontrado. 
   Además, la función también utiliza la función "busca2" para verificar si la 
   palabra contiene el carácter "~". Si lo encuentra, la función divide la palabra en 
   dos elementos utilizando el carácter "~" como separador y los almacena en un 
   arreglo llamado "elmtos". 
   Después de procesar todas las palabras en el arreglo "frase", la función llama a 
   otras funciones (Rellenar, Snippet y sugerencia) y asigna los resultados a 
   variables correspondientes (snippet y Sugerencia). 
   Finalmente, la función itera a través de un diccionario llamado "Palabra" y 
   verifica si todas las palabras en el diccionario están presentes en una lista 
   llamada "TodasPalabrasDeLosDocumentosSinRepetir". Si alguna palabra no 
   está presente, la función llama a la función "sugerencia" para sugerir una 
   corrección ortográfica para esa palabra y asigna el resultado a la variable 
   "Sugerencia". 
   Por último, la función crea un arreglo de objetos "SearchItem" utilizando los 
   resultados de la función "scoreOrdenado", y los devuelve como un objeto 
   SearchResult junto con la variable "Sugerencia".

\end{itemize}

\section{Propiedades y limitaciones del proyecto }
\begin{itemize}
    \item El usuario puede buscar no solo una palabra sino en general una frase 
    cualquiera. 
    \item La consulta del usuario puede contener símbolos que no sean solo letras y 
    números. 
    \item Se pueden introducir operadores en las consultas, tales como: 
    \item Se dará sugerencia al usuario a cerca de la consulta. 
    \item Está diseñado para trabajar con un vocabulario en español. 
    \item Para mayor eficiencia se sugiere que los archivos .txt no sean de gran 
    extensión. 
\end{itemize}




\end{document}